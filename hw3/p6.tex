\documentclass[12pt]{article}
\usepackage{fullpage,enumitem,amsmath,amssymb,graphicx}
\usepackage{fancyhdr}
\pagestyle{fancy}
\setlength{\headheight}{15pt}
\setlength{\headsep}{15pt}
\renewcommand{\headrulewidth}{0.4pt}
\lhead{Problem 3-6}
\chead{}
\rhead{Nisha Masharani (nisham)}
\lfoot{}
\cfoot{}
\rfoot{}
\begin{document}


\begin{center}
{\Large CS161 Homework 3 Problem 3-6}

\end{center}

\begin{enumerate}[label=(\alph*)]
  \item 
  \begin{eqnarray*}
  X_i &=& \text{the number of probes required for the ith insertion}\\
  A_j &=& \left\{\begin{array}{ll}1  & \mbox{if } j\text{th slot is occupied} \\ 0 & \text{otherwise}\end{array}\right.\\
  \Pr(A_j = 1) &=& \frac{i-1}{m}\\
  \Pr(X_i = k) &=& \left( \frac{i-1}{m} \right)^{k-1}\left( 1 - \frac{i-1}{m} \right)\\
  \Pr(X_i > k) &=& 1 - \Pr(X_i \le k)\\
  &=& 1 - \sum_{j = 1}^k \left( \frac{i-1}{m} \right)^{j-1}\left( 1 - \frac{i-1}{m} \right)\\
   &=& 1 - \left(1 -\left( \frac{i-1}{m} \right)^{k}\right)\\
   &=& \left( \frac{i-1}{m} \right)^{k}\\
   &\le& \left( \frac{n-1}{m} \right)^{k}\\
   &\le& \left( \frac{(m/2)-1}{m} \right)^{k}\\
   &=& \left( \frac{m-2}{2m} \right)^{k}\\
   &\le& \left( \frac{m}{2m} \right)^{k}\\
   &=& \left( \frac{1}{2} \right)^{k}\\
  \end{eqnarray*}
  Thus, $\Pr(X_i > k) \le \left( \frac{1}{2} \right)^{k}$.
  \item From part a, we have $P(X_i > k) \le \left( \frac{1}{2} \right)^{k}$. Plug in $k = 2\log n$ and we get:
  \begin{eqnarray*}
  \Pr(X_i > k) &\le& \left( \frac{1}{2} \right)^{k}\\
  \Pr(X_i > 2\log n) &\le& \left( \frac{1}{2} \right)^{2\log n}\\
  &=& 2^{-2\log n}\\
  &=& 2^{\log {n^{-2}}}\\
  &=& n^{-2}\\
  \end{eqnarray*}
  Thus, we have that $\frac{1}{n^2}$ is an upper bound for $P(X_i > 2\log n)$, so $P(X_i > 2\log n) = O\left(\frac{1}{n^2}\right)$
  \item
  \begin{eqnarray*}
  \Pr(X > k) &=& \Pr(X_1 > k) \text{or} \Pr(X_2 > k) \text{or} \Pr(X_3 > k) \text{or} ... \text{or} \Pr(X_n > k)\\
  &=& \Pr(X_1 > k) + \Pr(X_2 > k) + \Pr(X_3 > k) + ... + \Pr(X_n > k)\\
  &=& \sum_{i=1}^n \Pr(X_i > k)\\
  \end{eqnarray*}
  In this case, $k = 2\log n$, so we have:
  \begin{eqnarray*}
  \Pr(X > 2\log n) &=& \sum_{i=1}^n \Pr(X_i > 2\log n)\\
  &\le& \sum_{i=1}^n \frac{1}{n^2}\\
  &=& n \left(\frac{1}{n^2}\right)\\
  &=& \frac{1}{n}\\
  \end{eqnarray*}
  Thus, $\frac{1}{n}$ is an upper bound for $\Pr(X > 2\log n)$, so $\Pr(X > 2\log n) = O\left(\frac{1}{n}\right)$.
  \item We know $\Pr(X > 2\log n) = \frac{c}{n}$, where $c$ is some constant. That means $\Pr(X <= 2\log n) = 1 - \frac{c}{n}$.
  \begin{eqnarray*} 
  E[X] &=& \sum_x x\Pr(X = x)\\
  &\le& \left(\max_{x > 2\log n} x\right) \frac{c}{n} + \sum_{x <= 2\log n} x\Pr(X = x)\\
  &\le& \left(\max_{x > 2\log n} x\right) \frac{c}{n} + \left(\max_{x <= 2\log n} x\right) \left(1 -\frac{c}{n}\right) \\
  &=& n \frac{c}{n} + (2 \log n)\left(1 -\frac{c}{n}\right)\\
  &=& n \frac{c}{n} + (2 \log n)\left(\frac{n-c}{n}\right)\\
  &=& c + (2 \log n)\left(\frac{n-c}{n}\right)\\
  &\le& c + (2 \log n)\\
  \end{eqnarray*}

  Thus, $E[X] = O(\log n)$.

\end{enumerate}

\end{document}
