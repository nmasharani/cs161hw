\documentclass[12pt]{article}
\usepackage{fullpage,enumitem,amsmath,amssymb,graphicx}
\usepackage{fancyhdr}
\pagestyle{fancy}
\setlength{\headheight}{15pt}
\setlength{\headsep}{15pt}
\renewcommand{\headrulewidth}{0.4pt}
\lhead{}
\chead{}
\rhead{Nisha Masharani (nisham)}
\lfoot{}
\cfoot{}
\rfoot{}
\begin{document}


\begin{center}
{\Large CS161 Homework 5 Problem 5-5}

\end{center}
\section*{Algorithm}
Let us use dynamic programming. Our input sequence is a series of numbers stored in a 0-indexed array $A$. Our subproblem $C(i)$ is the maximum consecutive sum that includes the $i$th element in the sequence from indices $0$ to $i$. We can define $C(i)$ in terms of $C(i-1)$ as follows:\\
$C(i) = max(C(i-1) + A[i], A[i])$\\
Let us create another array $B$, of the same length as $A$. Iterate through $A$, calculating $B[i] = max(B[i-1] + A[i], A[i])$. Then, find the maximum value in $B$, and return that as the maximum sum. 

\section*{Correctness}
We need to prove that $C(i)$ gives the maximum consecutive sum that ends at i. Then, we can easily show that, if we calculate $C(i)$ for all $i$, and then return the maximum $C(i)$, we will end up with the maximum consecutive sum.\\
Let us examine the possible cases for the maximum consecutive sum ending at $i$:\\
\textbf{Case 1: The maximum consecutive sum starts at some element $j \ne i$ and ends at $i$.} In this case, the sum from $j$ to $i-1$ is the maximum sum that ends at $i-1$. To see this we can use a contradiction: assume that the maximum sum $C(i-1)$ that ends at $i-1$ is not the sum from $j$ to $i-1$, which we will denote as $k$. That means $C(i-1) > k$, so $C(i-1) + A[i] > k + A[i]$, but we said that $k + A[i]$ is the maximum consecutive sum that ends at $i$, so we have reached a contradiction and our original assumption cannot be true. Therefore, $C(i) = C(i-1) + A[i]$.\\
\textbf{Case 2: The maximum consecutive sum doesn't include any elements before $i$.} In this case, $C(i) = A[i]$ by definition.\\
In both of these cases, we get the maximum consecutive sum ending at $i$.\\
In our algorithm, we consider the maximum consecutive sums ending at all $i$, and take the maximum. Therefore, we correctly get the maximum consecutive sum in the entire sequence.

\section*{Running time}
To create $B$, we iterate over $A$ once, and calculate $B[i]$ from $B[i-1]$ and $A[i]$, which takes constant time, because $B[i-1]$ and $A[i]$ can both be accessed in constant time. Therefore, creating $B$ takes $\Theta(n)$ time, where $n$ is the length of the sequence of numbers. Then, we find the maximum value in $B$, which can be done by iterating over $B$ once, which also takes $\Theta(n)$ time. Therefore, the total running time is $\Theta(n)$.
\end{document}
