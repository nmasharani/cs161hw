\documentclass[12pt]{article}
\usepackage{fullpage,enumitem,amsmath,amssymb,graphicx}
\usepackage{fancyhdr}
\pagestyle{fancy}
\setlength{\headheight}{15pt}
\setlength{\headsep}{15pt}
\renewcommand{\headrulewidth}{0.4pt}
\lhead{}
\chead{}
\rhead{Nisha Masharani (nisham)}
\lfoot{}
\cfoot{}
\rfoot{}
\begin{document}


\begin{center}
{\Large CS161 Homework 5 Problem 5-2}

\end{center}

\section*{(1)}
% Use induction
We can use a cut and paste approach to prove that $G'$ is connected. Let $T_3 = (V, F_3)$ and $T_4 = (V, F_4)$ be nodes in $G'$, and let $S = F_4-F_3$ be the set of all edges that are in $T_4$ but not in $T_3$. Let us use the following cut and paste approach: \\
Remove an edge $uv$ from $T_3$ where $uv \notin F_4$. Removing this edge will give us two distinct "blobs", $A$ and $B$. $A$ and $B$ cannot be connected because there are no parallel edges in $T_3$, because there are no cycles in a spanning tree (proved Lecture 15 p172). Since $T_4$ is also a valid spanning tree, there must be an edge $u'v'$ in $S$ that connects a node $m \in A$ and $n \in B$, because there must be a subtree of $T_4$ that contains the same nodes as $A$ and another subtree that contains the same nodes as $B$, because $A + B = V$. Therefore, we can replace $uv$ with $u'v'$ and still have a valid spanning tree. Therefore, this new spanning tree, $T_3'$, must also be in $G'$. $T_3'$ only differs from $T_3$ by one edge, so there must be an edge $K$ in $G'$ from $T_3$ to $T_3'$.\\
By induction, we can repeat this cut and paste routine on $T_3'$ with all of the edges in $S$ until $S$ is empty. We know that the edge $K$ exists from $T_3$ to $T_3'$ will exist for all iterations of cut and paste, so the set of all $K$s is the path from $T_3$ to $T_4$. Since we repeat this routine $|T_4-T_3|$ times, the number of $K$s in that set must also be $|T_4-T_3|$.

\section*{(2)}
\subsection*{Algorithm}
Build two spanning trees, $T_{min}$ and $T_{max}$, where $T_{min}$ has the minimum number of blue edges possible, and $T_{max}$ has the maximum number of blue edges possible. To build $T_{min}$, set the weight of all blue edges to $1$ and the weight of all red edges to $0$, then use Kruskal's minimum spanning tree algorithm to get the spanning tree with the minimum cost, meaning the fewest blue edges. Keep track of the total cost $Cost(min)$ as the tree is built, and use that total cost as the number of blue edges in $T_{min}$. Build $T_{max}$ by setting the weight of all blue edges to $0$ and the weight of all red edges to $1$, and then using Kruskal's algorithm and keeping track of the cost similarly. Then, the number of blue edges is $V-Cost(max)-1$, because the total number of edges in the tree is $V-1$, and the cost gives the number of red edges in the tree.\\
If $k > V-Cost(max)-1$ or $k < Cost(min)$, then there is no valid spanning tree with k blue edges. Otherwise, use the following procedure:\\
Let $c$ be the number of blue edges in the current tree, $T$. Set $T = T_{min}$ and $c = Cost(min)$. Then, one at a time, iterate through the edges in $T_{max}$ that are not in $T_{min}$. For each edge, add the edge to $T$. Then, run DFS on one of the endpoints of the added edge until you find a back edge, which indicates that there is a cycle. Go through the cycle, and remove the first edge in the cycle that is not in $T_{max}$. If the added edge is blue, increment $c$. Continue this procedure until $c=k$.

\subsection*{Correctness}
We know that Kruskal's algorithm will give us the correct spanning trees with the minimum and maximum number of blue edges, so we now have to prove that we will get a tree with $k$ blue edges. We know from part one that if we have two spanning trees, in this case $T_{min}$ and $T_{max}$, there exists a path between the two trees such that at each step along the path, one edge is switched out for another. Since there are $Cost(min)$ blue edges in $T_{min}$ and $V-Cost(max)-1$ blue edges in $T_{max}$, and $k$ is between those two values, there must be at least one tree along that path that has $k$ blue edges, because the number of blue edges in the tree at each node along that path increases at most by one per edge. Therefore, if we just traverse that path, and keep track of the number of blue edges as we go, we can find a tree with $k$ blue edges.\\
We now must prove that our replacement procedure creates a valid spanning tree at each step. By induction. \textbf{Base case:} we start with $T = T_{min}$. Since $T_{min}$ was found with Kruskal's algorithm, and we know Kruskal's algorithm gives a valid spanning tree, $T$ is a valid spanning tree. \textbf{Inductive step:} Assume $T$ is a valid spanning tree. If we add in an edge from $T_{max}$ that is not in $T_{min}$, we must create a cycle, because adding any new edge to a spanning tree creates a cycle. We know DFS correctly finds cycles. Removing an edge from a cycle removes the cycle, so if we remove a single edge from our cycle, we have a tree that still connects all nodes, because the edge that we removed connected two nodes that were already in the tree. Therefore, we end up with a valid spanning tree at each step.

\subsection*{Running time}
$T_{min}$ and $T_{max}$ each take $O(E \log V)$ time to create using Kruskal's algorithm. The number of edges that differ between $T_{min}$ and $T_{max}$ is at most $V-1$, so in our replacement procedure, we will iterate through at most $V-1$ edges. Then, for each of those edges, we must remove the cycle, which can contain at most $V$ edges. Therefore, the total running time for the replacement procedure is $\Theta(V^2)$. Thus, the total running time for the algorithm is $O(E \log V + V^2)$.
\end{document}
