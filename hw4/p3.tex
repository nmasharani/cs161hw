\documentclass[12pt]{article}
\usepackage{fullpage,enumitem,amsmath,amssymb,graphicx}
\usepackage{fancyhdr}
\pagestyle{fancy}
\setlength{\headheight}{15pt}
\setlength{\headsep}{15pt}
\renewcommand{\headrulewidth}{0.4pt}
\lhead{}
\chead{}
\rhead{Nisha Masharani (nisham)}
\lfoot{}
\cfoot{}
\rfoot{}
\begin{document}


\begin{center}
{\Large CS161 Homework 4 Problem 4-3}

\end{center}

The algorithm from class does not correctly solve this problem. For example, let us consider the following problem:

\begin{verbatim}
Notation: (start, end) = revenue
1. (1, 2) = 1
2. (3, 4) = 1
3. (5, 6) = 1
4. (1, 7) = 10
\end{verbatim}

The algorithm from class would choose activities 1, 2, and 3, because activity 1 has the earliest end time. Then, after the choice of activity 1, activity 2 has the earliest end time. Then, after the choice of activities 1 and 2, activity 3 has the earliest end time. However, the total revenue of activities 1, 2, and 3 is 3, while the total revenue of selecting activity 4 only is 10. 
\end{document}
