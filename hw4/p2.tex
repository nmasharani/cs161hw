\documentclass[12pt]{article}
\usepackage{fullpage,enumitem,amsmath,amssymb,graphicx}
\usepackage{fancyhdr}
\pagestyle{fancy}
\setlength{\headheight}{15pt}
\setlength{\headsep}{15pt}
\renewcommand{\headrulewidth}{0.4pt}
\lhead{}
\chead{}
\rhead{Nisha Masharani (nisham)}
\lfoot{}
\cfoot{}
\rfoot{}
\begin{document}


\begin{center}
{\Large CS161 Homework 4 Problem 4-2}

\end{center}

\section*{Algorithm}
Sort all of the lectures in increasing order by end time using merge sort, where if two items have the same end time, the item with the smaller start time goes first. Starting at the beginning of the list, keep track of the lecture with the earliest end time, which has index i. Then, iterate through the list with index j, putting each lecture into a new lecture hall, until you find the first item in the list with a start time later than the earliest end time. Put that lecture in the same lecture hall as the lecture at index i. Then, increment both i and j and repeat the process until j reaches the end of the list of lectures.

\section*{Correctness}
To prove that this algorithm is correct, we must prove firstly that only one lecture takes place per lecture hall at a time, and secondly, that we use as few lecture halls as possible. \\
To prove that only one lecture takes place per lecture hall at a time, we can see from our algorithm that we only add an item to a lecture hall containing another item if its start time is after the end time of the previous lecture in that lecture hall. Therefore, there cannot be more than one lecture in a lecture hall at a time.\\
To prove that we use as few lecture halls as possible, we will show that for every lecture hall, we fit as many lectures in as possible. We will then show that if for every lecture hall we fit in as many lectures as possible, the total number of lecture halls must be minimal.\\
As proven in lecture, if we choose the next lecture for a specific lecture hall A to be the lecture $k$ with the earliest end time $f_k$ such that $s_k > f_j$, where $j$ is the last lecture to take place in the lecture hall, $k$ is the optimal choice to maximize the number of lectures that take place in the lecture hall.\\
We can extend this proof to consider multiple lecture halls as follows: let lecture $j$ be the lecture with the earliest end time, $f_j$, and let lecture $k$ be the lecture with the earliest end time $f_k$ with start time $s_k > f_j$. Let lecture $i$ be a lecture between lecture $j$ and lecture $k$. We know that lecture $i$ must have $f_i \ge f_j$, but we also know that lecture $i$ must have $s_i < f_j$. Therefore, all lectures $i$ must overlap with lecture $j$, so none of those lectures can be in the same lecture hall as lecture $j$. We can also show that all lectures $i$ must overlap with each other, because they must all have $s_i < f_j$ and $f_j \le f_i$, so all lectures $i$ must at least overlap during time $f_j$. Therefore, all lectures $i$ overlap, and cannot be in the same lecture hall. We proved in lecture that if we have a lecture $k$ with the smallest finish time $f_k$, there exists an optimal choice of non-overlapping lectures that contains $k$ as its first element. Therefore, since we similarly choose the first lecture with the smallest finish time, our choice is also optimal for all lecture halls.\\

\section*{Running time}
Running merge sort on the lectures takes $\Theta(n log n)$ time, where $n$ is the number of lectures. Then, because we only iterate through the list of lectures once to find the correct assignment, assuming that putting a lecture into a lecture hall takes $\Theta(1)$ time, determining the assignments takes $\Theta(n)$ time. Thus, the total running time of our algorithm is $\Theta(n log n)$.
\end{document}
