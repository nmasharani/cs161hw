\documentclass[12pt]{article}
\usepackage{fullpage,enumitem,amsmath,amssymb,graphicx}
\usepackage{fancyhdr}
\pagestyle{fancy}
\setlength{\headheight}{15pt}
\setlength{\headsep}{15pt}
\renewcommand{\headrulewidth}{0.4pt}
\lhead{}
\chead{}
\rhead{Nisha Masharani (nisham)}
\lfoot{}
\cfoot{}
\rfoot{}
\begin{document}


\begin{center}
{\Large CS161 Homework 4 Problem 4-1}

\end{center}
Let us consider the following tree:
\begin{verbatim}
            10
     5               15
 3       7       13      17
1 4     6 8    11  14  16  18
\end{verbatim}
Let us binary search in this tree for the value 11. In this case, our path will go through the following nodes: $10, 15, 13, 11$. Thus, $B = \{10, 15, 13, 11\}$. All of the nodes to the left of the path are $A = \{5, 3, 7, 1, 4, 6, 8\}$. All of the nodes to the right of the path are $C = \{14, 16, 17, 18\}$. We had claimed that for any keys $b \in B$ and $c \in C$, $b \le c$, but from this example, we can see that $15 \in B$ and $14 \in C$, so this claim cannot be true. 
\end{document}
