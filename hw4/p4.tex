\documentclass[12pt]{article}
\usepackage{fullpage,enumitem,amsmath,amssymb,graphicx}
\usepackage{fancyhdr}
\pagestyle{fancy}
\setlength{\headheight}{15pt}
\setlength{\headsep}{15pt}
\renewcommand{\headrulewidth}{0.4pt}
\lhead{}
\chead{}
\rhead{Nisha Masharani (nisham)}
\lfoot{}
\cfoot{}
\rfoot{}
\begin{document}


\begin{center}
{\Large CS161 Homework 4 Problem 4-4}

\end{center}

\section*{Algorithm}
Let $a_i$ be the last gas station at which we refueled, and let $a_j$ be the next gas station. Increment $j$ until we reach the first $a_j - a_i > R$. Add $a_{j-1}$ to our list of gas stations at which we refuel, and set $i = j-1$. Repeat until $j > n$.

\section*{Correctness}
Let $a_k$ be the distance from our starting point to the last gas station at which we refueled (assume we start with a full tank of gas at location 0, so $a_k = 0$ at the start), and let $S_k$ be the set of all distances for gas stations that come after $a_k$. $S$ is the set of all gas stations and $S_k = \{a_i \in S : i > k\}$. Let $A_k$ be a minimum length sequence of refueling points given that we refueled at $a_i$, and let $b_i$ be the refueling point selected by our algorithm. Let $a_j$ be the first refueling point in $A_k$. Since $A_k$ is a valid solution, we cannot run out of gas before reaching $a_j$ from $a_i$, so the $a_j - a_i < R$. Let us examine the cases for $a_j$:\\
\textbf{Case 1: $a_j = b_i$.} In this case, since $A_k$ is an optimal solution, we therefore know that $b_i$ is a correct choice that will lead to an optimal solution.\\
\textbf{Case 2: $a_j < b_i$.} In this case, since $b_i$ is further away from $a_i$ than $a_j$, and $b_i$ is also a valid choice of gas station, we can replace $a_j$ with $b_i$ in $A_k$ and still get a valid solution.\\
\textbf{Case 3: $a_j > b_i$.} This case is not possible, because the $b_i$ is the gas station furthest away from $a_i$ that can be reached with a full tank, so any stations that are futher away are not valid choices, and cannot be a valid solution.\\
Thus, our algorithm chooses a next gas station such that we get a minimal solution. Therefore, our algorithm produces a minimal solution overall.

\section*{Running Time}
We iterate through the list of gas stations once, which runs in $\Theta(n)$ time. Assuming that all comparisons and adding gas stations to our solution run in constant time, our total running time is therefore $\Theta(n)$.
\end{document}
