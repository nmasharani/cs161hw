\documentclass[12pt]{article}
\usepackage{fullpage,enumitem,amsmath,amssymb,graphicx}
\usepackage{fancyhdr}
\pagestyle{fancy}
\setlength{\headheight}{15pt}
\setlength{\headsep}{15pt}
\renewcommand{\headrulewidth}{0.4pt}
\lhead{}
\chead{}
\rhead{Nisha Masharani (nisham)}
\lfoot{}
\cfoot{}
\rfoot{}
\begin{document}


\begin{center}
{\Large CS161 Homework 4 Problem 4-5}

\end{center}
\section*{Data Structure}
Our data is a binary search tree with additional information stored at each node. At each node, in addition to storing the element as the key, also store the minimum value of MinDifference for all nodes in the subtree, including the current node. 

\section*{Insertion}
\textbf{Algorithm and Correctness:} To insert an element $k$, start by inserting it as if our tree were a normal binary search tree (described on slide 123). Then, find the predecessor and successor of $k$ using the normal binary search tree predecessor and successor methods (described on slide 125) and keep references $p$ and $s$, respectively, to the predecessor and successor. Let $m = \min(\{|p-k|, |s-k|\})$, and set the min value associated with $k$ to be m. This is the minimum distance associated with k, because predecessor and successor is known to correctly retrieve the closest elements above and below k. Traverse upwards in the binary tree from $k$ and repeatedly set the min value associated with the node to m until we reach the first node with min associated value $n$ such that $n < m$. We stop at this node because we want every node to have a min value associated with it that is the min value in the entire subtree under that node, including the other paths. If n is a min value at some node, that means that some other path, that we did not traverse upwards, has a min value of n, so the min value of the entire subtree is that value. Similarly, for the $p$ or $s$ that yielded the minimum distance from $k$, if $m < n$, where n is the minimum associated value, replace the min value with $m$ and iterate upwards until $n < m$.\\
\textbf{Running time:} Since our BST is balanced, insertion, successor, and predecessor all take $\Theta(\log n)$. Iterating upwards from $k$ takes at most $\Theta(\log n)$, because $k$ is at at most depth $\log n$ and can traverse backwards to at most the root, which takes $\Theta(\log n)$. Similarly, the predecessor/successor that yields the min distance is also at at most depth $\log n$, so traversing upwards from the predecessor/sucessor also takes $\Theta(\log n)$. Rotations also take $\Theta(\log(n))$.Thus, the total running time of insertion is $c \log n = \Theta(\log n)$.

\section*{Deletion}
\textbf{Algorithm and Correctness:} Let us examine the three cases for deleting $k$.\\
Case 1: k has no children. In this case, we can simply delete $k$ and make the pointer from the parent null, which we have proved works in lecture. Then, we must update the minimum values stored. To calculate the minimum distance to store at parent(k), we must look at the successor s and predecessor p of parent(k), and the minimum distance c stored at the other child node of parent(k) if parent(k) has another child node. Similarly to insertion, this is correct because predecessor and successor are defined to be the closest elements to k. Let $m = min(parent(k) - s, parent(k) - p, c)$. Replace the min value stored at parent(k), and then iterate upwards in the tree, replacing the minimum values along the way until we encounter a minimum value $n < m$. As proved for insertion, this is correct because we want the minimum distance in the entire subtree.\\
Case 2: k has 1 child. In this case, we can redirect the pointer from parent(k) to child(k), eliminating k from the tree. This is a valid delete operation as proved in lecture. In this case, we must also update the minimum values stored the same way we did in case 1, but with looking at both (if they exist) children of parent(k) instead of only the one remaining child (if it exists) in case 1.\\
Case 3: k has 2 children. In this case, we will recursively delete either the predecessor or the successor of k, and then will put that predecessor or successor in place of k. Similarly to cases 1 and 2, we know that this delete operation works as proved in lecture. In this case, we must update the minimum value of the value that replaces k in the same way that we updated the values in part 2 and in insert.\\
\textbf{Running time:} For all three cases, finding k takes $\Theta(\log(n))$. Bubbling up the minimum distance also takes at most $\Theta(\log(n))$, because the height of the tree is $\log n$ and we are at most traversing to the top of the tree once each for k, the predecessor, and the successor. In case 3, we make a recursive call, but that recursive call is guaranteed to reach one of the non-recursive cases, so the running time is still a constant factor multiple of $\log n$. Thus, the total running time is $\Theta(\log(n))$.\\

\section*{MinDistance}
\textbf{Algorithm:} Return the min value stored at the root node.\\
\textbf{Correctness:} We have proved that the minimum value is bubbled up correctly in insertion and deletion. Therefore, the minimum value stored at the root node must be the minimum distance in the entire tree.\\
\textbf{Running time:} The retrieval time is the amount of time it takes to access the root node, which is assumed to be constant. Therefore, the total running time is $\Theta(1)$.\\
\end{document}
